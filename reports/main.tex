%!TEX root = ../thesis.tex
%!TEX TS-program = lualatex
% vim:fenc=utf-8
% vim:fdm=marker



% preambulo latex 
\documentclass[11pt]{article}
\usepackage[spanish,es-lcroman,es-nodecimaldot]{babel}
\usepackage{amsmath,amsfonts,amssymb,units,xcolor}
\usepackage{graphicx,tabularx}
\usepackage{booktabs}
\usepackage{multirow}
\usepackage{fancyhdr}
\usepackage{hyperref}
\usepackage{blindtext}
\usepackage{lastpage}
\usepackage{fontspec}\usepackage{titlesec}
\usepackage[export]{adjustbox}
\usepackage[letterpaper,left=3cm,right=2cm,top=2cm,bottom=2cm]{geometry}
\usepackage[sort&compress,authoryear]{natbib}
\usepackage[justification=centerlast,labelfont=bf,font=scriptsize,
            margin=1.5cm,labelsep=period]{caption}


% definir variables
% -----------------
%\newcommand{\bomm}{BOMM1-ITS}


% configuracion de los paquetes
% -----------------------------

% configuracion de la fuente
\DeclareTextFontCommand{\emph}{\slshape}
\renewcommand\familydefault{\sfdefault}

% espaciado de las lineas para las págines iniciales
\renewcommand{\baselinestretch}{1.25} 
\setlength{\parindent}{0mm}
\setlength{\parskip}{2.5mm}

% hipervinculos
\hypersetup{%
    colorlinks,
    linkcolor={red!50!black},
    citecolor={blue!50!black},
    urlcolor={blue!80!black}
    }

% encabezado
\pagestyle{fancy}
\lhead{\includegraphics[height=0.75cm,valign=c]{./logo_cicese.jpg}\vspace{1pt}}
\chead{}
\rhead{Pág. \thepage/\pageref{LastPage}}
\lfoot{}
\cfoot{}
\rfoot{}
\renewcommand{\footrulewidth}{0.4pt}

% pagina de titulo
\title{
  \vspace{2em}
  \Large{\bfseries Reporte Técnico} \\
  \vspace{3em}
  \LARGE{\bfseries
    Base de datos de las observaciones de las Boyas Oceanográficas y de
    Meteorología Marina (BOMM)
  } \\
  \vspace{3em}
}
%
\author{
  \includegraphics[height=2cm,valign=c]{./logo_cicese.jpg} \\
  \vspace{0.5ex} \\
  Grupo de Oleaje \\
  \small{Departamento de Oceanografía Física} \\
  \small{Centro de Investigación Científica y de Educación Superior} \\
  \small{Ensenada, B.C., México} \\
  \vspace{2em} \\
}
\date{
  \today \\
  Revisión 1.0
  \vspace{2em} \\
  \begin{figure}
    \centering
    \includegraphics[height=2.1cm,valign=c]{./logo_cigom.jpeg}
    \hspace{0.5cm}
    \includegraphics[height=2.3cm,valign=c]{./logo_sener.jpg}
  \end{figure}
  \vfill
}


%  empezar documento    
% =============================================================================
\begin{document}

% crear titulo
\maketitle
\thispagestyle{empty}
\newpage
\thispagestyle{fancy}
\tableofcontents
\newpage

% introduccion {{{
\section{Introducción}
\label{sec:introduccion}

En este reporte técnico se presenta una descripción detallada del conjunto de
datos observados con las Boyas Oceanográficas y de Meterología Marina (BOMM).


% niveles de procesamiento {{{
\subsection{Niveles de procesamiento}
\label{sub:niveles_de_procesamiento}

Para los fines de este trabajo, se definieron cuatro niveles de procesamiento,
los cuales se describen a continuación:

\begin{description}
  \item[Nivel 0] En este nivel están los datos binarios que se escriben en la
    memoria interna de los instrumentos y los datos que son transmitidos a
    través de comunicación entre los instrumentos y la computadora de la boya.
    Estos datos están en formato ASCII. Estos datos no tienen ningún control de
    calidad ni tienen incorporados los metadatos.

  \item[Nivel 1] El primer nivel de procesamiento consiste en generar la base de
    datos en formato netCDF4 a partir de los datos de Nivel 0. Se hace una
    acomodación de los datos en una secuencia creciente del tiempo de acuerdo a
    su tasa de muestreo. Adicionalmente, se llenan los espacios vacíos con
    banderas de datos no válidos (NaN). Se incorporan los metadatos e
    información adicional para realizar correcciones y calibraciones.
    \textbf{Ejemplos:} \emph{aceleraciones y tasas de cambio de los ángulos en el
    marco de referencia del sensor de movimiento, velocidad del viento sin
    corregir por el movimiento de la boya, posición de los WaveStaff para el
    cálculo de los espectros direccionales.}

  \item[Nivel 2] En este nivel se presentan las variables derivadas a la misma
    tasa de muestreo del los datos del Nivel 1. Se aplican las correcciones y
    calibraciones necesarias. Se aplican un control de calidad de los
    datos de acuerdo a criterios físicos y estadísticos.
    \textbf{Ejemplos:} \emph{elevación de la superficie libre del mar, velocidad
    del viento corregida por el movimiento de la boya, perfil de corrientes
    superficiales en el marco de referencia inercial.}

  \item[Nivel 3] En el último nivel se presentan las variables resultantes de
    agrupar y combinar los datos en un tiempo determinado. Específicamente, se
    presentan promedios de las variables provenientes de diferentes sensores. Se
    aplican métodologías establecidas para el cálculo de variables derivadas.
    \textbf{Ejemplos:} \emph{espectros direccionales, esfuerzos de Reynolds, rapidez y
    dirección del viento, parametros integrales del oleaje, etc.}

\end{description}

% }}}

% archivos en netcdf {{{
\subsection{Formato NetCDF}
\label{sub:formato_netcdf}

Según el SMID de CIGOM (\url{http://smid.cigom.org/smid-docs/}), en la Línea 1
del proyecto se estableció que la entrega de conjuntos de datos será en formato
NetCDF ya que es un formato ampliamente utilizado por la comunidad científica y
además es compatible con una gran variedad de software de análisis y
visualización. NetCDF (\emph{Network Common Data Form}) es un conjunto de
bibliotecas de software y estándares de formato de datos, el cual es
independiente de la arquitectura de la máquina, es de código abierto, y permite
la creación, acceso y distribución de datos cientificos orientados a arreglos.
Una de sus mayores ventajas es que es autodescriptivo, es decir, los archivos
contienen los datos y los metadatos. Otra de sus ventajas es la portabilidad y
la escalabilidad, que permiten una fácil distribución y el acceso a subconjuntos
más pequeños de los datos que están alojados en un servidor remoto.
Específicamente la versión 4 del formato NetCDF permite la separación de los
datos en grupos, lo cual es muy útil cuando se trabajan con datos de diferentes
sensores y medidos a diferentes tasas de muestreo en el mismo dataset, como es
el caso de los datos de las boyas oceanográficas.


% }}}

% }}}

% Procesamiento de los datos {{{
\section{Descripción de la base de datos}%
\label{sec:descripcion_de_la_base_de_datos}

% datos de nivel 0 {{{
\subsection{Nivel 0} 
\label{sub:nivel_0}

Se consideran como datos crudos los datos que se escriben directamente en el
disco duro de la boya. Estos datos no tiene ningún tipo de procesamiento, y se
escriben tal cual como salen de cada uno de los sensores.  Los datos crudos son
los datos de nivel 0 y tienen la siguiente estructura:

\begin{verbatim}
  data/
  |-- level0/
  |   |-- acelerometro/
  |   |-- anemometro/
  |   |-- binary/
  |   |-- gps/
  |   |-- marvi/
  |   |-- maximet/
  |   |-- msg/
  |   |-- proceanus/
  |   |-- rbr/
  |   |-- signature/
  |   |-- vector/
  |   `-- wstaff/
  |-- level1/
  |-- level2/
  `-- level3/
\end{verbatim}

Se genera una carpeta por cada sensor. En la mayoría de los sensores (los de alta
frecuencia) se escribe una carpeta por año, una por mes, una por por día y una por
hora; y se escribe un archivo cada 10 minutos. Por ejemplo, la ruta del archivo que
corresponde a las 10:20 am del 5 de enero de 2018 del anemómetro sónico es:

\begin{verbatim}
  ./data/level0/anemometro/2018/01/05/10/anemometro-1801051020.csv
\end{verbatim}

Por otra lado, los sensores de baja frecuencia, es decir, los que reportan
promedios de los datos, como el sensor de CO2 y el CTD, se escribe una carpeta
por año y una carpeta por mes, y se escribe un archivo por día, el cual contiene
los datos a la tasa de muestreo específica.

En la carpeta \texttt{binary} se almacenan los archivos binarios originales de
los sensores que así lo permiten. Por ejmplo, en el caso de la BOMM1-ITS que
estuvo midiendo cerca de la Isla de Todos Santos, entre noviembre de 2017 y
febrero de 2018, los sensores que permitieron almacenar los datos binarios
fueron el acelerómetro (Ekinox2-M), el MARVI (Módulo de Adquisición y Regulación
de Voltaje Inteligente) y el velocímetro Vector. En la carpeta \texttt{msg} se
almacenan los promedios de los datos que son enviados vía satélite. En las
carpetas \texttt{level1}, \texttt{level2} y \texttt{level3} se almacenan los
datos procesados en los diferentes niveles, los cuales se describen en las
siguientes secciones.
% }}}

% datos de nivel 1 {{{
\subsection{Nivel 1}
\label{sub:nivel_1}

\subsubsection*{Análisis de los datos}

Se realizó un análisis y procesamiento de los datos crudos para convertirlos al
formato netCDF. Para esto, primero se hace una acomodación de los datos en una
secuencia creciente de tiempo de acuerdo a su tasa de muestreo. Por ejemplo,
algunos sensores presentan imprecisiones en su reloj interno, lo que implica que
la tasa de muestreo a la que se programan no sea constante en el tiempo lo cual
genera pequeñas variaciones, que hacen que en ocasiones se tengan más o menos
datos de los que se esperan en cierto intervalo de tiempo. Por ejemplo, los
alambres de capacitancia se programan para medir a una tasa de muestreo de 20
datos por segundo, pero debido a la deriva del reloj, en ocasiones se tiene 19 o
21 datos en un segundo. Para solucionar esto, se genera una arreglo del tiempo
de acuerdo a la tasa de muestreo del instrumento y se aplica una interpolación
lineal de los datos. Si hay más del 10\% de datos inválidos dentro de un
intervalo de 10 minutos, ese intervalo se considera como inválido. Cuando son
menos de 10\% los datos inválidos y estos no están de forma continua en el
tiempo, los datos inválidos se reemplazan por promedios del intervalo de 10
minutos con el fin de aplicar la interpolación.  Finalmente, cuando se tiene
espacios vacíos, es decir, intervalos de tiempo donde la boya no registró datos,
se genera un arreglo con datos nó válidos (NaN) del mismo tamaño y se llenan
dichos espacios vacíos. En los datos de Nivel 1 se eliminan también los datos en
los que la boya no estaba instalada. Estos datos si se conservan en el Nivel 0.

\subsubsection*{Estructura de los archivos}%

Los datos de nivel 1 de procesamiento se escriben en archivos netCDF4, usando la
capacidad de generar grupos de datos, característica de este formato. Se genera
un archivo netCDF4 por cada día de datos. El nombre cada archivo consiste en la
fecha en el formato \texttt{'yyyymmdd'} más la extensión \text{'.nc'}, por
ejemplo:

\begin{verbatim}
  data/
  `-- level1/
      |-- 20171117.nc
      |-- 20171118.nc
      |-- 20171119.nc
      |--     .
      |--     .
      |--     .
      |-- 20180131.nc
      `-- 20180201.nc
\end{verbatim}


Cada archivo agrupa los datos en grupos por cada sensor. Por ejemplo, los datos
de la BOMM1-ITS se clasifican en los siguientes grupos.

\begin{verbatim}
  groups: ekinox, sonic, gps, marvi, maximet, proceanus,
          rbr, signature, vector, wstaff
\end{verbatim}

Los grupos contienen las variables y las dimensiones. En este caso solo se tiene
dos tipos de dimensiones, el tiempo y el número de celdas, este último es
específico del perfilador de corrientes Signature 1000 kHz. El siguiente es un
ejemplo de la estructura del grupo asociado con los datos del sensor de
movimiento Ekinox2-M:

\begin{verbatim}
  group /ekinox:
      sampling_frequency: 100
      serial_number: 5242914
      description: MRU Subsea Ekinox2-M
      convention: X positive towards north buoy, Y eastward and Z downward
      dimensions(sizes): time(8640000)
      variables(dimensions): float64 time(time), float64 accel_x(time),
                             float64 accel_y(time), float64 accel_z(time),
                             float64 gyro_x(time), float64 gyro_y(time),
                             float64 gyro_z(time), float64 delta_vel_x(time),
                             float64 delta_vel_y(time), float64 delta_vel_z(time),
                             float64 delta_ang_x(time), float64 delta_ang_y(time),
                             float64 delta_ang_z(time), float64 temp(time)
\end{verbatim}

Cada una de las variables tiene atributos, para los cuales se sigue la
convención CF-1.7 y las recomendaciones de los manuales de usuario de cada uno
de los instrumentos. Por ejemplo, los atributos de la tasa de cambio del ángulo
al rededor del eje $x$, se presentan a continuación:

\begin{verbatim}
  float64 gyro_x(time)
      _FillValue: nan
      standard_name: gyro_x
      long_name: rate of change of the angle in X direction
      units: rad/s
  path = /ekinox
  unlimited dimensions:
  current shape = (8640000,)
  filling on
\end{verbatim}

Los metadatos de las mediciones que se incorporan en el archivo netCDF se
presentan en un archivo aparte con un formato YAML (\url{http://yaml.org/}) ya
que es el formato más amigable para este tipo de información. Por ejemplo, los
atributos globales que identifican la información de la boya son:

\begin{verbatim}
title: >
  This dataset presents air-sea physical and chemical variables from a BOMM
  (Oceanographic and Marine Meteorology buoy) near the Isla Todos Santos,
  Ensenada, BC, México, from Nov 2017 to Jan 2018.

history: >
  The BOMM1-ITS was deployed on 2017/11/16 and recovered on 2018/02/02.

source: Ocean surface observations
convention: CF-1.7
institution: CICESE - CIGOM
creator_name: The waves group - CICESE
creator_url: https://www.cicese.mx/
email: oleaje@cicese.mx
acknowledgments: >
  This research has been funded by Fondo Sectorial CONACYT-SENER
  Hidrocarburos, Project 201441.

comments: |
  Sampling frequency is shown in Hz, in all cases.
\end{verbatim}

Igualmente, en el formato YAML se presentan los metadatos asociados con los
atributos de cada variable. Para esto se separan en grupos de acuerdo con cada
sensor como se muestra a continuación:

\begin{verbatim}
  ekinox:
    sampling_frequency: 100
    seconds_per_file: 600
    serial_number : 024000042
    description: MRU Subsea Ekinox2-M
    convention: X positive towards north buoy, Y eastward and Z downward
    variables:
      accel_x:
        column: 6
        long_name: acceleration in X direction
        units: m/s^2
      accel_y:
        column: 7
        long_name: acceleration in Y direction
        units: m/s^2
        .
        .
        .
      delta_ang_z:
        column: 18
        long_name: coning output in Z direction
        units: rad/s
      temp:
        column: 15
        long_name: sensor temperature
        units: degrees_celsius
\end{verbatim}

En este caso, los metadatos son los mismos que se escribieron en el archivo
NetCDF, excepto por algunos como \texttt{column} que es el número de la columna
en la que se presentan los datos en el archivo ASCII del nivel 0.

\subsubsection*{Ejemplos de acceso a los datos}
\label{ssub:ejemplos_de_acceso_a_los_datos}

Los archivos netCDF4 son de fácil acceso en la mayoría de los lenguajes de
programación, por ejemplo, en Python se usa la paquetería
\texttt{netcdf4-python} (\url{http://unidata.github.io/netcdf4-python/}). A
continuación se presenta un ejemplo simple de cómo leer los datos de nivel 1 en
\texttt{python3.6.5}.

\begin{verbatim}
  import numpy as np
  import netCDF4 as nc
  #
  filename = "../data/20171117.nc" 
  data = nc.Dataset(filename) #<-- carga el dataset completo
  ekx = data["ekinox"]        #<-- lee el grupo del ekinox
  met = data["maximet"]       #<-- lee el grupo de la maximet
  #
  a, b = 0, 180000            #<-- indices para el tiempo 30 mins
  ax = ekx["accel_x"][a:b]    #<-- extrae la aceleracion en x
\end{verbatim}



% }}}

% datos de nivel 2 {{{
\subsection{Nivel 2}
\label{sub:nivel_2}
TODO
% }}}

% datos de nivel 3 {{{
\subsection{Nivel 3}
\label{sub:nivel_3}
TODO
% }}}


% }}}


\end{document}
%  finalizar documento --------------------------------------------------------
